\section{The Pahlavi Continuity}

\begin{center}
\textit{``I wish I could show you,\\
when you are lonely or in darkness,\\
the astonishing light of your own being.''}\\
\vspace{0.2cm}
--- Hafez (Khwaja Shams-ud-Din Muhammad Hafez-e Shirazi)
\end{center}

\vspace{0.5cm}

\subsection{From Cyrus to Reza: 2,500 Years of Persian Leadership}

The name Cyrus---Kourosh---has echoed through Persian history for 25 centuries. It is no accident that Iran's crown prince bears this legacy.

\begin{center}
\textbf{The Line of Persian Greatness:}
\end{center}

\begin{tcolorbox}[colback=persiangold!10!white,colframe=cyruscolor,width=\textwidth,arc=2mm,boxrule=1pt]
\begin{center}
\textbf{Cyrus the Great} (559--530 BCE)\\
\textit{Founded the Persian Empire, authored the first human rights charter}\\
$\downarrow$\\
\textbf{The Achaemenid Dynasty} (550--330 BCE)\\
\textit{Built the largest empire the ancient world had seen}\\
$\downarrow$\\
\textbf{The Sasanian Empire} (224--651 CE)\\
\textit{Revived Persian greatness, preserved Zoroastrian culture}\\
$\downarrow$\\
\textbf{The Pahlavi Dynasty} (1925--1979)\\
\textit{Modernized Iran, celebrated Cyrus at Persepolis}\\
$\downarrow$\\
\textbf{Reza Pahlavi} (Crown Prince)\\
\textit{Carries the torch of secular, democratic Iran}
\end{center}
\end{tcolorbox}

\subsection{The 2,500 Year Celebration}

In 1971, Mohammad Reza Shah Pahlavi hosted the world at Persepolis for the 2,500th anniversary of the Persian Empire. At the tomb of Cyrus the Great in Pasargadae, he spoke words that still resonate:

\begin{quote}
\textit{``Cyrus, great King, King of Kings, Achaemenian King, King of the land of Iran. I, the Shahanshah of Iran, offer thee salutations from myself and from my nation. Rest in peace, for we are awake, and we will always stay awake.''}
\end{quote}

Today, his son \textbf{Reza Pahlavi} continues this sacred duty---not as a would-be monarch, but as a voice for democratic transition, secular governance, and national unity. He has explicitly stated he seeks no throne, only the freedom of his people.

\subsection{A Secular, Democratic Iran}

The CYRUS token and its community stand for the same principles Reza Pahlavi has championed:

\begin{itemize}[leftmargin=*]
    \item \textbf{Secularism} --- Separation of religion and state, as Cyrus practiced
    \item \textbf{Democracy} --- Government by consent of the governed
    \item \textbf{Human Rights} --- The Cyrus Cylinder as our founding document
    \item \textbf{National Unity} --- Persians, Kurds, Azeris, Baluch---all children of Iran
    \item \textbf{Territorial Integrity} --- One Iran, undivided
    \item \textbf{Women's Equality} --- The rights Cyrus granted restored and expanded
\end{itemize}

We do not seek to impose any leader. We seek to empower the Iranian people to choose their own future. But we believe that future must be rooted in the values of Cyrus---the values that made Persia great.
