\section{CYRUS Protocol: Post-Quantum Secure Communications}

\begin{center}
\textit{``Raise your words, not your voice.\\
It is rain that grows flowers, not thunder.''}\\
\vspace{0.2cm}
--- Rumi
\end{center}

\vspace{0.5cm}

Liberation requires coordination. Coordination requires communication. And communication under surveillance is not communication---it is entrapment.

The CYRUS community recommends and supports the development of post-quantum secure messaging protocols that protect activists, journalists, and citizens from both current surveillance and future quantum attacks. This section specifies the technical requirements and architecture.

\subsection{Threat Model}

We assume adversaries with:
\begin{itemize}[leftmargin=*]
    \item \textbf{Mass surveillance capability}: Ability to intercept and store all network traffic
    \item \textbf{State-level resources}: Access to nation-state cryptanalytic capabilities
    \item \textbf{Future quantum computers}: Ability to break classical asymmetric cryptography (RSA, ECDH) within 10--20 years
    \item \textbf{Endpoint compromise attempts}: Malware, physical seizure, coercion
    \item \textbf{Network-level attacks}: Traffic analysis, timing attacks, node compromise
\end{itemize}

\subsection{Security Goals}

The protocol must achieve:

\begin{table}[h]
\centering
\begin{tabular}{@{}ll@{}}
\toprule
\textbf{Property} & \textbf{Description} \\ \midrule
Confidentiality & Only intended recipients can read messages \\
Authenticity & Recipients can verify sender identity \\
Forward Secrecy & Past messages safe even if keys later compromised \\
Post-Compromise Security & Recovery after temporary key compromise \\
Metadata Protection & Sender/recipient relationship hidden from observers \\
Post-Quantum Resistance & Secure against quantum cryptanalysis \\
Offline Capability & Messages deliverable when recipient offline \\
Multi-Device Support & Seamless use across multiple devices \\
Decentralization & No single point of failure or control \\
\bottomrule
\end{tabular}
\caption{CYRUS Protocol Security Requirements}
\end{table}

\subsection{Cryptographic Primitives}

\subsubsection{Hybrid Post-Quantum Key Exchange}

Classical elliptic curve cryptography (X25519) is efficient but vulnerable to Shor's algorithm on quantum computers. Post-quantum lattice-based schemes (ML-KEM/CRYSTALS-Kyber) resist quantum attacks but are newer and less battle-tested.

\textbf{Solution}: Hybrid key exchange combining both:

\begin{equation}
K_{\text{shared}} = \text{KDF}(K_{\text{X25519}} \| K_{\text{ML-KEM}})
\end{equation}

Security holds unless \textit{both} schemes are broken---classical cryptanalysis defeats ML-KEM \textit{and} quantum computers defeat X25519 simultaneously.

\begin{table}[h]
\centering
\begin{tabular}{@{}lll@{}}
\toprule
\textbf{Primitive} & \textbf{Algorithm} & \textbf{Purpose} \\ \midrule
Classical KEM & X25519 & ECDH key agreement \\
Post-Quantum KEM & ML-KEM-768 (Kyber) & Quantum-resistant key encapsulation \\
Symmetric Encryption & ChaCha20-Poly1305 & AEAD for message encryption \\
Key Derivation & HKDF-SHA256 & Derive session keys from shared secrets \\
Signatures & Ed25519 + Dilithium & Authentication (hybrid PQ) \\
Hash Function & SHA-256/SHA-3 & Integrity, ratchet advancement \\
\bottomrule
\end{tabular}
\caption{Cryptographic Primitives}
\end{table}

\subsubsection{Double Ratchet with PQ Enhancement}

The Double Ratchet algorithm (Signal Protocol) provides forward secrecy and post-compromise security. We enhance it with post-quantum key encapsulation:

\begin{enumerate}[leftmargin=*]
    \item \textbf{Root Key Ratchet}: Updated with each new DH/KEM exchange
    \item \textbf{Sending/Receiving Chains}: Derived from root key, advanced per-message
    \item \textbf{Message Keys}: Ephemeral, deleted immediately after use
    \item \textbf{PQ Ratchet}: Periodic ML-KEM re-keying for quantum resistance
\end{enumerate}

\begin{figure}[h]
\centering
\begin{tikzpicture}[scale=0.9]
    % Root chain
    \node[draw, rounded corners, fill=persianblue!20, minimum width=2cm, minimum height=0.8cm] (rk1) at (0,3) {$RK_0$};
    \node[draw, rounded corners, fill=persianblue!20, minimum width=2cm, minimum height=0.8cm] (rk2) at (3,3) {$RK_1$};
    \node[draw, rounded corners, fill=persianblue!20, minimum width=2cm, minimum height=0.8cm] (rk3) at (6,3) {$RK_2$};

    \draw[->, thick] (rk1) -- (rk2) node[midway, above] {\small DH+KEM};
    \draw[->, thick] (rk2) -- (rk3) node[midway, above] {\small DH+KEM};

    % Sending chains
    \node[draw, rounded corners, fill=persiangold!30, minimum width=1.5cm, minimum height=0.6cm] (ck1) at (0,1.5) {$CK_0$};
    \node[draw, rounded corners, fill=persiangold!30, minimum width=1.5cm, minimum height=0.6cm] (ck2) at (1.5,1.5) {$CK_1$};
    \node[draw, rounded corners, fill=persiangold!30, minimum width=1.5cm, minimum height=0.6cm] (ck3) at (3,1.5) {$CK_2$};

    \draw[->, thick] (rk1) -- (ck1);
    \draw[->, thick] (ck1) -- (ck2);
    \draw[->, thick] (ck2) -- (ck3);

    % Message keys
    \node[draw, rounded corners, fill=persianred!20, minimum width=1cm, minimum height=0.5cm] (mk1) at (0,0) {$MK$};
    \node[draw, rounded corners, fill=persianred!20, minimum width=1cm, minimum height=0.5cm] (mk2) at (1.5,0) {$MK$};
    \node[draw, rounded corners, fill=persianred!20, minimum width=1cm, minimum height=0.5cm] (mk3) at (3,0) {$MK$};

    \draw[->, thick] (ck1) -- (mk1);
    \draw[->, thick] (ck2) -- (mk2);
    \draw[->, thick] (ck3) -- (mk3);

    % Labels
    \node at (-2,3) {\small Root Chain};
    \node at (-2,1.5) {\small Chain Keys};
    \node at (-2,0) {\small Message Keys};
\end{tikzpicture}
\caption{Post-Quantum Enhanced Double Ratchet}
\end{figure}

\subsection{Anonymous Transport: Onion Routing}

Messages are wrapped in multiple encryption layers and routed through a network of nodes, so no single node knows both sender and recipient.

\subsubsection{Three-Hop Onion Requests}

\begin{enumerate}[leftmargin=*]
    \item \textbf{Entry Node}: Knows sender IP, but not destination or content
    \item \textbf{Middle Node}: Knows only previous and next hop
    \item \textbf{Exit Node}: Knows destination swarm, but not sender
\end{enumerate}

Each hop strips one encryption layer:
\begin{equation}
\text{Onion} = E_{K_1}(\text{addr}_2 \| E_{K_2}(\text{addr}_3 \| E_{K_3}(\text{swarm} \| E_{\text{recipient}}(\text{message}))))
\end{equation}

\subsubsection{Decentralized Message Storage (Swarm)}

Messages are stored in recipient's ``swarm''---a set of nodes determined by a DHT (Distributed Hash Table) based on recipient's public key. This enables:
\begin{itemize}[leftmargin=*]
    \item \textbf{Offline delivery}: Messages persist until recipient retrieves them
    \item \textbf{Redundancy}: Multiple nodes store each message
    \item \textbf{Censorship resistance}: No single node can block delivery
    \item \textbf{Expiration}: Messages auto-delete after configurable period
\end{itemize}

\subsection{Multi-Device Architecture}

Users need messaging across phone, laptop, tablet. The protocol supports this via:

\begin{itemize}[leftmargin=*]
    \item \textbf{Per-Account Keys}: Master identity key pair (long-term)
    \item \textbf{Per-Device Keys}: Each device has unique keys linked to account
    \item \textbf{Config Messages}: Encrypted sync of device list, settings, contacts
    \item \textbf{Explicit Device Management}: Users can add/remove devices, revoke compromised ones
\end{itemize}

\subsection{Implementation Recommendations}

The CYRUS community endorses and recommends:

\begin{itemize}[leftmargin=*]
    \item \textbf{Session Messenger}: Open-source, implements this architecture, actively developing Protocol V2 with PQ support
    \item \textbf{Signal Protocol}: Excellent ratchet design, adding PQ (PQXDH)
    \item \textbf{Matrix/Element}: Decentralized, end-to-end encrypted, federation model
\end{itemize}

\textbf{For activists inside Iran}: Use Session or Signal over Tor/VPN. Assume all unencrypted communications are monitored. Delete messages after reading. Use disappearing messages. Never discuss operational details over any electronic medium unless absolutely necessary.

\subsection{The Mathematics of Freedom}

Cryptography is mathematics in service of liberty. The security of these protocols rests on problems believed to be computationally hard:

\begin{itemize}[leftmargin=*]
    \item \textbf{Discrete Logarithm Problem}: Given $g^x \mod p$, find $x$ (classical security)
    \item \textbf{Elliptic Curve DLP}: Given $[k]P$ on curve $E$, find $k$ (compact classical security)
    \item \textbf{Learning With Errors (LWE)}: Given $(\mathbf{A}, \mathbf{As} + \mathbf{e})$, find $\mathbf{s}$ (post-quantum security)
    \item \textbf{Module-LWE}: Structured variant enabling efficient ML-KEM
\end{itemize}

These mathematical structures---groups, rings, lattices---are the foundation of secure communication. Every time an activist sends an encrypted message, they are protected by millennia of mathematical discovery, from Euclid to Diffie-Hellman to the lattice cryptographers of today.

\textbf{This is why we call on mathematicians to join our cause.} Your work is not abstract---it is the armor of the oppressed.
